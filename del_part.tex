 \deleted{Thousands of exoplanets have been discovered with photometric transit surveys from the ground (e.g., HATnet \citep{2004PASP..116..266B}, SuperWASP \citep{2006PASP..118.1407P}, KELT \citep{2007PASP..119..923P}) and from space (CoRoT \citep{2009A&A...506..411A}, Kepler \citep{2010Sci...327..977B}, K2 \citep{2014PASP..126..398H}). Future planet hunting surveys such as TESS \citep{2014SPIE.9143E..20R}, CHEOPS \citep{2017SPIE10563E..1LC} and PLATO \citep{2014ExA....38..249R} aim to sample brighter stars at faster cadences over larger field of views. The detection of Earth analogues (Earth-sized planets in the habitable zone of KGF-type stars) with the transit method, however, is very difficult because of their small transit depth ($\approx100\,$ppm) which is at the current limit of photometric surveys \citep{2008A&A...485..607C} and is similar to the average stellar variability \citep{2002ESASP.485...35B,2013ApJ...769...37B,2015ApJ...810...29H}. In addition, the periods of these planets, of order one year, allow only for the stacking of a few transits for typical mission lifetimes.}
 
 \explain{Dein Einstieg zum Paper ist auf jeden Fall lesenwert und auch in vielen Papers \"ahnlich zu finden. Wir k\"onnen ihn so nehmen und das w\"are von mir aus okay. Allerdings wirkt der Einstieg so auch nicht sehr packend, gerade weil man ihn so schon 1000mal gelesen hat. Was h\"altst Du davon, wenn wir aber mal was anderes versuchen und den Leser gleich am Anfang des Papers packen und an den Kern unserer Fragestellung f\"uhren? (Wei\ss \, nicht, ob mir das hier besonders gelungen ist, vielleicht kannst Du noch mehr auf die Tube dr\"ucken.) F\"ur meinen Vorschlag habe ich einiges von Dir \"ubernommen.}
 
 \deleted{After a signal is discovered, its transit duration and depth estimates can be improved with a trapezoid fit \citep{2016A&C....17....1H}.}
 
 
\footnote{The Optical Gravitational Lensing Experiment (OGLE) contributed some of the first transiting exoplanets as serendipitous discoveries \citep{2003Natur.421..507K}.}.

\deleted{The rich datasets of photometric timeseries are most commonly searched for exoplanet transit signatures using the Box Least Squares (BLS) algorithm \citep{2002A&A...391..369K,2016ascl.soft07008K}. It assumes that the light curve has two discrete values, high (out of transit) and low (in transit). Thus, it searches the data for box-shaped (rectangular) features. A true transit signal, however, exhibits a gradual ingress and egress and a trough-shape around mid-transit, caused by limb-darkening and mostly influenced by the planet-to-star radius ratio, and the planetary impact parameter. The difference between the true transit shape and the box produces extra noise in the test statistic, and dilutes the true signal. Reducing this noise by fitting for a better shape than a box allows for a better trade-off between false alarms and missed detections, and therefore increases the planet detection yield at a given signal-to-noise threshold.}

%The problem of transit search is inherently parallel as there are many potential host stars on a CCD sensor. Within a light curve, parallelization is easily possible as many trial periods have to be searched. GPUs are useful and BLS implementations for GPUs exist, e.g. the \texttt{cuvarbase} package\footnote{\url{https://johnh2o2.github.io/cuvarbase/}}.

%As in every signal search in noisy data, the search efficiency is a trade-off between false alarms and missed detections. Both elements can be visualized as overlapping Gaussian distributions. For the BLS algorithm, the false alarms arise from noise in the data, and from a mismatch between the true shape of a (transit) signal and the search filter (a box). Reducing this part of the noise will therefore reduce false alarms, and therefore increase detection efficiency at a given signal-to-noise threshold. In other words, searching data for rectangular transits is less efficient than searching the data for a more appropriate signal shape. Taken to an extreme, consider a BLS-like algorithm searching for parallelogram-shaped features. It might be more sensitive for finding planets with large eccentricity (as these tend to show asymmetric transit signatures), but less efficient for all other transits. This is because of the larger deviation between the shapes of a parallelogram and a planetary transit. Clearly, fitting for the right signal increases detection efficiency. We will quantify the improvement in Section~X.

\deleted{
Such a model comparison is performed by the Hunt for Exomoons with Kepler \citep{2012ApJ...750..115K}, where (known) transit signatures are tested for transit versus transit-with-moon models. While this approach uses additional parameters for the putative exomoon, it takes the planetary transit locations as already known. Its computational effort is extremly large, of order $33{,}000$ CPU hours per light curve \citep{2015ApJ...813...14K}.}




- show that transits become more box-like with increasing wavelength


Let $d$ be the best-fitting transit depth, so that $d$ is the deepest depth of the transit. For gaussian noise, the common statistic for the significance of a transit is the signal to noise ratio at depth $d$ \citep[Equation 2 in][]{2006MNRAS.373..231P}

\begin{equation}
S_d = \frac{d}{\sigma_o} n^{1/2}
\end{equation}

In a box-shaped model, all transit points are expected to occur, on average, at the bottom level. In our transit-shaped model, each point has a depth $\leq d$. For $b=0$, $u=0.5$, the relative integral of all points $c = d_{\LD}/d \sim 0.88$. For grazing transits, $c \rightarrow 0$. This factor is required to approximate the (averaged) true depth, so that

\begin{equation}
S_{\rm LD} = \frac{c d}{\sigma_o} n^{1/2}
\end{equation}


TTV+TDV broaden/smear a transit, making it less box-like
get transit durations https://exoplanetarchive.ipac.caltech.edu/cgi-bin/TblView/nph-tblView?app=ExoTbls&config=cumulative

get ttv scatter per planet (defined as 1.4826 times the median absolute deviation of TTV from its median) \citep{2016ApJS..225....9H}

calculate ttv/duration, make histogram

From visual inspection, morphological differences become relevant for TTV scatter per transit duration of larger than 20\,\%. Only 5\,\% of known transiting exoplanets have such large TTV scatter.

 Problems:
 - Eccentricity?

 
 \deleted{The ``duty cycle'', i.e. the transit duration per period, has similar bounds as the period, due to physical constraints. The search edges are, however, more complex if eccentricity is considered. It is insightful to examine the transit duration of the known transiting exoplanets, as a fraction of their period (Figure~\ref{fig:per_t14}). The commonly used BLS search grid encompasses about $10^{-4}\lesssim T_{14}/P \lesssim 0.3$. However, more than half of this region is not populated. For example, a transiting planet with $P=1\,$d and $T_{14}/P=10^{-3}$ would require an extreme eccentricity. While such cases might not be physically impossible, they are certainly extremely rare. For most planet searches, which do not aim to examine completeness explicitly, they can be ignored. Again, our TLS algorithm takes a user-defined duration grid as an input parameter. If none is provided, it defaults to the region indicated with dashed lines in Figure~\ref{fig:per_t14}.}
